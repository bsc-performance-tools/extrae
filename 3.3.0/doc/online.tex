\chapter{Extrae On-line User Guide}

\section{Introduction}

Extrae On-line is a new module developed for the \TRACE tracing toolkit, available from 
version 3.0, that incorporates intelligent monitoring, analysis and selection of the traced data. This 
tracing setup is tailored towards long executions that are producing large traces. Applying automatic 
analysis techniques based on clustering, signal processing and active monitoring, Extrae gains the 
ability to inspect and filter the data while it is being collected to minimize the amount of data 
emitted into the trace, while maximizing the amount of relevant information presented. 

Extrae On-line has been developed on top of Synapse, a framework 
that facilitates the deployment of applications that follow the master/slave architecture based on the 
MRNet software overlay network. Thanks to its modular design, new types of automatic 
analyses can be added very easily as new plug-ins into the on-line tracing system, just by defining 
new Synapse protocols.

This document briefly describes the main features of the Extrae On-line module, and shows how it has 
to be configured and the different options available.


\section{Automatic analyses}

\TRACE On-line currently supports three types of automatic analyses: fine-grain structure detection 
based on clustering techniques, periodicity detection based on signal processing techniques, and 
multi-experiment analysis based on active monitoring techniques. \TRACE On-line has to be 
configured to apply one of these types of analyses, and then the analysis will be performed 
periodically as new data is being traced. 


\subsection{Structure detection}

This mechanism aims at identifying the fine-grain structure of the computing regions of the program. 
Applying density-based clustering, this method is able to expose the main performance trends in the 
computations, and this information is useful to focus the analysis on the zones of real interest. 
To perform the cluster analysis, \TRACE On-line relies on the ClusteringSuite tool\footnote{You can download it from \url{http://www.bsc.es/computer-sciences/performance-tools/downloads}.}.

At each phase of analysis, several outputs are produced:

\begin{itemize}
 \item A scatter-plot representation that illustrates the behavior of the main computing regions of the program, 
that enables a quick evaluation of potential imbalances.
 \item A summary of several performance metrics per cluster.
 \item On supported machines, a CPI stack model that attributes stall cycles to specific hardware components.
 \item And a trace that is augmented with the clusters information, that allows to identify 
patterns of performance and variabilites. 
\end{itemize}

Subsequent clustering results can be used to study the evolution of the application over time. In order
to study how the clusters are evolving, the xtrack tool can be used.


\subsection{Periodicity detection}

This mechanism allows to detect iterative patterns over a wide region of time, and precisely delimit where the 
iterations start. Once a period has been found, those iterations presenting less perturbations are selected to 
produce a representative trace, and the rest of the data is basically discarded. The result of applying this 
mechanism is a compact trace where only the representative iterations are traced in full detail, and for the 
rest of the execution we can optionally keep summarized information in the form of phase profiles or 
a ``low resolution'' trace. 

Please note that applying this technique to a very short execution, or if no periodicity can be detected in the 
application, may result in an empty trace depending on the configuration options selected (see Section 
\ref{sec:Configuration}).


\subsection{Multi-experiment analysis}

This mechanism employs active measurement techniques in order to simulate different execution scenarios 
under the same execution. \TRACE On-line is able to add controlled 
interference into the program to simulate different computation loads, network bandwidth, memory congestion
and even tuning some configurations of the parallel runtime (currently supports MPI Dynamic Load Balance (DLB)
runtime). Then, the application behavior can be studied under different circumstances, and tracking can be 
used to analyze the impact of these configurations on the program performance.
This technique aims at reducing the number of executions necessary to evaluate different parameters and 
characteristics of your program.


\section{Configuration}
\label{sec:Configuration}

  In order to activate the On-line tracing mode, the user has to enable the corresponding configuration section 
  in the \TRACE XML configuration file. This section is 
  found under \emph{trace-control > remote-control > online}. The default configuration is already 
  ready to use:  

\begin{verbatim}
<online enabled="yes" 
        analysis="clustering" 
        frequency="auto" 
        topology="auto">
\end{verbatim}

  The available options for the <online> section are the following:
  
\begin{itemize}
 \item \textbf{enabled}: Set to ``yes'' to activate the On-line tracing mode.
 \item \textbf{analysis}: Choose from ``clustering'', ``spectral'' and ``gremlins''.
 \item \textbf{frequency}: Set the time in seconds after which a new phase of analysis will be triggered, or ``auto'' to let \TRACE decide this automatically.
 \item \textbf{topology}: Set the desired tree process tree topology, or ``auto'' to let \TRACE decide this automatically.
\end{itemize}

  Depending on the analysis selected, the following specific options become available.

\subsection{Clustering analysis options}

\begin{verbatim}
<clustering config="cl.I.IPC.xml"/>
\end{verbatim}

\begin{itemize}
 \item \textbf{config}: Specify the path to the ClusteringSuite XML configuration file.
\end{itemize}

\subsection{Spectral analysis options}

\begin{verbatim}
<spectral max_periods="0" num_iters="3" min_seen="0" min_likeness="85">
   <spectral_advanced enabled="no" burst_threshold="80">
      <periodic_zone     detail_level="profile"/>                 
      <non_periodic_zone detail_level="bursts" min_duration="3s"/> 
   </spectral_advanced>
</spectral>
\end{verbatim}

The basic configuration options for the spectral analysis are the following:

\begin{itemize}
 \item \textbf{max\_periods}: Set to the maximum number of periods to trace, or ``all'' to explore the whole run.
 \item \textbf{num\_iters}: Set to the number of iterations to trace per period.
 \item \textbf{min\_seen}: Minimum repetitions of a period before tracing it (0 to trace the first time that you encounter it)
 \item \textbf{min\_likeness}: Minimum percentage of similarity to compare two periods equivalent.
 \item \textbf{min\_likeness}: Minimum percentage of similarity to compare two periods equivalent.
\end{itemize}

Also, some advanced settings are tunable in the <spectral\_advanced> section:

\begin{itemize}
 \item \textbf{enabled}: Set to ``yes'' to activate the spectral analysis advanced options.
 \item \textbf{burst\_threshold}: Filter threshold to keep all CPU bursts that add up to the given total time percentage.
 \item \textbf{detail\_level}: Specify the granularity of the data stored for the non-representative iterations of the 
 periodic region, and in the non-periodic regions. Choose from none (everything is discarded), 
 profile (phase profile at the start of each iteration/region) or bursts (trace in bursts mode).
 \item \textbf{min\_duration}: Minimum duration in seconds of the non-periodic regions for emitting any information 
 regarding that region into the trace. 
\end{itemize}
           
\subsection{Gremlins analysis options}

\begin{verbatim}
<gremlins start="0" increment="2" roundtrip="no" loop="no"/>
\end{verbatim}

\begin{itemize} 
 \item \textbf{start}: Number of gremlins at the beginning of the execution.
 \item \textbf{increment}: Number of extra gremlins at each analysis phase. Can also be a negative value to indicate that
 you want to remove gremlins.
 \item \textbf{roundtrip}: Set to ``yes'' if you want to start adding gremlins after 
 you decrease to 0, or vice-versa, start removing gremlins after you reach the maximum.
 \item \textbf{loop}: Set to ``yes'' if you want to go back to the initial number 
 of gremlins and repeat the sequence of adding/removing gremlins after you have finished a complete sequence.
\end{itemize}
        



