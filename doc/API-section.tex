\chapter{\TRACE API}\label{cha:API}

\section{Basic API}\label{sec:BasicAPI}

The following routines are defined in the {\tt \$\{EXTRAE\_HOME\}/include/extrae\_user\_events.h}.

\begin{itemize}

 \item {\tt void Extrae\_init (void)}\\
 Initializes the tracing library.\\
 {\bf NOTE:} This routine is called automatically by {\tt MPI\_Init} and when tracing is performed by the DynInst launcher.

 \item {\tt void Extrae\_fini (void)}\\
 Finalizes the tracing library and dumps the intermediate tracing buffers onto disk.\\
 {\bf NOTE:} This routine is called automatically by {\tt MPI\_Finalize} and when the tracing is performed by the DynInst launcher.

 \item {\tt void Extrae\_event (int event, int value)}\\
 The Extrae\_event adds a single timestamped event into the tracefile. The event has two arguments: type and value.

 Some common use of events are:
  \begin{itemize}
   \item Identify loop iterations (or any code block): Given a loop, the user can set a unique type for the loop and a value related to the iterator value of the loop. For example:
    \begin{verbatim}
     for (i = 1; i <= MAX_LOOP; i++)
     {
       Extrae_event (1000, i);
       [original loop code]
     }
     Extrae_event (1000, 0);
    \end{verbatim}
   The last added call to Extrae\_event marks the end of the loop setting the event value to 0, which facilitates the analysis with Paraver.
   \item Identify user routines: Choosing a constant type (6000019 is usual in our tracing tools) and different values for different routines (set to 0 to mark a "leave" event) 
    \begin{verbatim}
     void routine1 (void)
     {
      Extrae_event (6000019, 1);
      [routine 1 code]
      Extrae_event (6000019, 0);
     }

     void routine2 (void)
     {
      Extrae_event (6000019, 2);
      [routine 2 code]
      Extrae_event (6000019, 0);
     }
   \end{verbatim}
   \item Identify any point in the application using a unique combination of type and value.
  \end{itemize}

 \item {\tt void Extrae\_nevent (unsigned int count, unsigned int *types, unsigned int *values)}\\
  Allows the user to place {\em count} events with the same timestamp at the given position.

 \item {\tt void Extrae\_counters (void)}\\
  Emits the value of the active hardware counters set. See chapter \ref{cha:XML} for further information.

 \item {\tt void Extrae\_eventandcounters (int event, int value)}\\
  This routine lets the user add an event and obtain the performance counters with one call and a single timestamp.

 \item {\tt void Extrae\_neventandcounters (int event, int value)}\\
  This routine lets the user add several events and obtain the performance counters with one call and a single timestamp.

 \item {\tt void Extrae\_shutdown (void)}\\
  Turns off the instrumentation.

 \item {\tt void Extrae\_restart (void)}\\
  Turns on the instrumentation.

 \item {\tt void Extrae\_previous\_hwc\_set (void)}\\
  Makes the previous hardware counter set defined in the XML file to be the active set (see section \ref{sec:XMLSectionMPI} for further information).

 \item {\tt void Extrae\_next\_hwc\_set (void)}\\
  Makes the following hardware counter set defined in the XML file to be the active set (see section \ref{sec:XMLSectionMPI} for further information).

 \item {\tt void Extrae\_set\_tracing\_tasks (int from, int to)}\\
  Allows the user to choose from which tasks (not {\em threads}!) store informartion in the tracefile

 \item {\tt void Extrae\_set\_options (int options)}\\
  Permits configuring several tracing options at runtime. The {\tt options} parameter has to be a bitwise or combination of the following options, depending on the user's needs:
  \begin{itemize}
   \item {\tt EXTRAE\_CALLER\_OPTION}\\
    Dumps caller information at each entry or exit point of the MPI routines. Caller levels need to be configured at XML (see chapter \ref{cha:XML}).
   \item {\tt EXTRAE\_HWC\_OPTION}\\
    Activates hardware counter gathering.
   \item {\tt EXTRAE\_MPI\_OPTION}\\
    Activates tracing of MPI calls.
   \item {\tt EXTRAE\_MPI\_HWC\_OPTION}\\
    Activates hardware counter gathering in MPI routines.
   \item {\tt EXTRAE\_OMP\_OPTION}\\
    Activates tracing of OpenMP runtime or outlined routines.
   \item {\tt EXTRAE\_OMP\_HWC\_OPTION}\\
    Activates hardware counter gathering in OpenMP runtime or outlined routines.
   \item {\tt EXTRAE\_UF\_HWC\_OPTION}\\
    Activates hardware counter gathering in the user functions.
  \end{itemize}

 \item {\tt void Extrae\_network\_counters (void)}\\
  Emits the value of the network counters if the system has this capability. {\em (Only available for systems with Myrinet GM/MX networks).}

 \item {\tt void Extrae\_network\_routes (int task)}\\
  Emits the network routes for an specific {\tt task}. {\em (Only available for systems with Myrinet GM/MX networks).}

\end{itemize}

\section{Special considerations for Cell Broadband Engine tracing package}

Instead of including {\tt \$\{EXTRAE\_HOME\}/include/extrae\_user\_events.h} include:
\begin{itemize}
 \item {\tt \$\{EXTRAE\_HOME\}/include/ppu\_trace\_sdk2.h} on the PPE side, and,
 \item {\tt \$\{EXTRAE\_HOME\}/include/sputrace\_user\_events.h} on the SPE side.
\end{itemize}

\subsection{PPE side}\label{subsec:PPEside}

The routines shown on section \ref{sec:BasicAPI} are available for the PPE element. In addition, two additional routines are available to control the creation and finalization of the SPE threads. These routines are:

\begin{itemize}

 \item {\tt int  CELLtrace\_init (int spus, spe\_context\_ptr\_t * spe\_ids)}\\
 Contacts with the SPE thread to initialize once the SPE tracing environment. Such call has to be synchronized with the invocation of {\tt SPUtrace\_init} (see \ref{subsec:SPEside}) call on the SPE side due to the presence of message passing using the mailboxes. The routine receives the total number of contexts created by the Cell SDK {\tt spe\_context\_create} and a vector pointing to those contexts. Each of those contexts will reference to a single SPE thread created by a call to {\tt pthread\_create}.

 \item {\tt void CELLtrace\_fini (void)}\\
 Waits for the finalization of all the threads registered in {\tt CELLtrace\_init} and dumps their intermediate tracing buffers.

\end{itemize}

\subsection{SPE side}\label{subsec:SPEside}

Due to the lack of parallel paradigms and hardware counters inside the SPE element, the SPE tracing library is a subset of the typical tracing library. The following API calls are available for the SPE element:

\begin{itemize}

 \item {\tt void SPUtrace\_init (void)}\\
 Initializes the tracing package in the SPE side. It has to be synchronized with {\tt CELLtrace\_init} due to the message passing using mailboxes.

 \item {\tt void SPUtrace\_fini (void)}\\
 Notifies the finalization of the work performed in the SPE thread and transfers the tracing buffer to the PPE element.

 \item {\tt void SPUtrace\_event (int event, int value)}\\
 Has the same semantics as {\tt Extrae\_event}.

 \item {\tt void SPUtrace\_nevent (unsigned int count, unsigned int *types, unsigned int *values)}\\
 Has the same semantics as {\tt Extrae\_nevent}.

\end{itemize}

